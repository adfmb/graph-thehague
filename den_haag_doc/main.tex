%%%%%%%%%%%%%%%%%%%%%%%%%%%%%%%%%%%%%%%%%%%%%%%%%%%%%%%%%%%%%%%%%%%%%%%%%%%%%%%%
%2345678901234567890123456789012345678901234567890123456789012345678901234567890
%        1         2         3         4         5         6         7         8

\documentclass[letterpaper, 10 pt, conference]{ieeeconf}  % Comment this line out
                                                          % if you need a4paper
%\documentclass[a4paper, 10pt, conference]{ieeeconf}      % Use this line for a4
                                                          % paper

\IEEEoverridecommandlockouts                              % This command is only
                                                          % needed if you want to
                                                          % use the \thanks command
\overrideIEEEmargins
% See the \addtolength command later in the file to balance the column lengths
% on the last page of the document



% The following packages can be found on http:\\www.ctan.org
%\usepackage{graphics} % for pdf, bitmapped graphics files
%\usepackage{epsfig} % for postscript graphics files
%\usepackage{mathptmx} % assumes new font selection scheme installed
%\usepackage{times} % assumes new font selection scheme installed
%\usepackage{amsmath} % assumes amsmath package installed
%\usepackage{amssymb}  % assumes amsmath package installed

\title{\LARGE \bf
Crimen en La Haya, Pa\'ises Bajos BLA BLA
}

%\author{ \parbox{3 in}{\centering Huibert Kwakernaak*
%         \thanks{*Use the $\backslash$thanks command to put information here}\\
%         Faculty of Electrical Engineering, Mathematics and Computer Science\\
%         University of Twente\\
%         7500 AE Enschede, The Netherlands\\
%         {\tt\small h.kwakernaak@autsubmit.com}}
%         \hspace*{ 0.5 in}
%         \parbox{3 in}{ \centering Pradeep Misra**
%         \thanks{**The footnote marks may be inserted manually}\\
%        Department of Electrical Engineering \\
%         Wright State University\\
%         Dayton, OH 45435, USA\\
%         {\tt\small pmisra@cs.wright.edu}}
%}

\author{Pedro Vladimir Hern\'andez Serrano, Jos\'e Alfredo M\'endez Barrera y Elisa Hern\'andez Rodr\'iguez % <-this % stops a space
}


\begin{document}



\maketitle
\thispagestyle{empty}
\pagestyle{empty}


%%%%%%%%%%%%%%%%%%%%%%%%%%%%%%%%%%%%%%%%%%%%%%%%%%%%%%%%%%%%%%%%%%%%%%%%%%%%%%%%
\begin{abstract}

Este documento se analiz\'o

\end{abstract}


%%%%%%%%%%%%%%%%%%%%%%%%%%%%%%%%%%%%%%%%%%%%%%%%%%%%%%%%%%%%%%%%%%%%%%%%%%%%%%%%
\section{INTRODUCI\'ON}

La descomposi\'on de las colonias conlleva consecuencias que afectan el nivel de vida de las personas e indirectamente aquejan a colonias cercanas. Una de las situaciones m\'as frecuentes por las cuales surge dicha descomposici\'on  son la delincuencia y el miedo o percepci\'on hacia la delincuencia, lo que puede desencadenar la reubicación como una medida de precaución. En este contexto, surge una pregunta fundamental; ¿Las personas se reubican en colonias con menor delincuencia?

\section{DATOS}

\subsection{Descripci\'on de los datos}

En este an\'alisis se contruy\'o una base de datos de panel, considerando  las variables que pueden estar ligadas con la reubicaci\'on (migraci\'on interna). La base de datos se compone de los 114 colonias que constituyen la Haya en los Pa\'ises Bajos desde 2003 hasta 2009 (las cifras se ha extra\'ido de los documentos oficiales recopilados por el municipio de La Haya (Den Haag in cijfers, DHIC). La base de datos incluye diez grupos de variables: 

\begin{enumerate}
\item \textit{\textbf{Flujos de reubicaci\'on}}
Los fujos de reubicaci\'on incluyen 114 colonias en la Haya, est\'a informaci\'on est\'a disponible para 7 años (2003-2009).
\item \textit{\textbf{Crimen}}
Estas variables se refieren a la informaci\'on reportada al Departamento de Polic\'ia. Por su naturaleza, las variables se dividieron en en dos grupos: 
\begin{itemize}
\item Crimen cometido directamente a personas (o crimen por violencia). Este tipo crimen incluye amenazas, maltrato y robo en las calles.
\item Crimen a propiedad, el cual se compone po las variables robo de autom\'ovil, robo de objetos del autom\'ovil, robo en compañ\'ias o negocios, robo de mercanc\'ia en tiendas, robo de cartera y robo en casa-habitaci\'on.
\end{itemize}

\item \textit{\textbf{Atracciones}}
Las atracciones consideradas

\item \textit{\textbf{Educaci\'on}}

\item \textit{\textbf{Empleo}}

\item \textit{\textbf{Poblaci\'on}}

\item \textit{\textbf{Bienes ra\'ices}}

\item \textit{\textbf{Distancia}}

\item \textit{\textbf{Composici\'on \'Etnica}}

\item \textit{\textbf{Ingreso}}

\end{enumerate}

\subsection{Metodolog\'ia}

¿Las personas se reubican en colonias con menor delincuencia?

BLABLA

Se utilizar\'a 

\section{RESULTADOS}

aqui el o los grafos



\section{DISCUSI\'ON}

El an\'alisis realizado puede ser extendido ....



%%%%%%%%%%%%%%%%%%%%%%%%%%%%%%%%%%%%%%%%%%%%%%%%%%%%%%%%%%%%%%%%%%%%%%%%%%%%%%%%
\section*{ANEXOS}

Appendixes should appear before the acknowledgment.



%%%%%%%%%%%%%%%%%%%%%%%%%%%%%%%%%%%%%%%%%%%%%%%%%%%%%%%%%%%%%%%%%%%%%%%%%%%%%%%%




\begin{thebibliography}{10}

\bibitem{c1} G. O. Young, ÒSynthetic structure of industrial plastics (Book style with paper title and editor),Ó 	in Plastics, 2nd ed. vol. 3, J. Peters, Ed.  New York: McGraw-Hill, 1964, pp. 15Ð64.
\bibitem{c2} W.-K. Chen, Linear Networks and Systems (Book style).	Belmont, CA: Wadsworth, 1993, pp. 123Ð135.
\bibitem{c3} H. Poor, An Introduction to Signal Detection and Estimation.   New York: Springer-Verlag, 1985, ch. 4.
\bibitem{c4} B. Smith, ÒAn approach to graphs of linear forms (Unpublished work style),Ó unpublished.

\end{thebibliography}




\end{document}
